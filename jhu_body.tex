\begin{frame}[t]{Introduction}
	{Background}
	
	\begin{itemize}
		\item <1-> Major theoretical work on unsteady aerodynamics dates back to 1930s
		\begin{itemize}
			\item <1-> \textit{H. Glauert (1930) The force and moment on an oscillating airfoil.}
			\item <1-> \textit{T. Theodorsen (1935) General theory of aerodynamic instability and the mechanism of flutter.}
			\item <1-> \textit{K\'{a}rm\'{a}n \& Sears (1938) Airfoil theory for non-uniform motion.}		
		\end{itemize}
		\item <1-> Mathematical modeling for flutter prediction was laid down by 1940.
		\begin{itemize}
			\item <1-> Has worked very well in unsteady aerodynamics.
		\end{itemize}
	\item <2-> Two salient features:
		\begin{itemize}
			\item <2-> Inviscid and Linear theories.
		\end{itemize}
	\end{itemize}

	\onslide<3-> Doctoral work has been in flow regimes where the Classical aerodynamic theory starts to fail.
	
	\begin{tikzpicture}[remember picture,overlay]
	\node (sw) at (page cs:0,0){}; % bottom left
	\node (se) at (page cs:1,0){}; % bottom right
	\node (ne) at (page cs:1,1){}; % top    right
	\node (nw) at (page cs:0,1){}; % top    left
	
	\end{tikzpicture}
	
\end{frame}
%-------------------------------------------------------------------------------

\begin{frame}[t]{Introduction}
	{Transitional $Re$}
	
	\begin{itemize}
		\item <1-> Aeroelasticity at low Reynolds numbers.
		\begin{itemize}
			\item <1-> \textit{Poirel et al. (2008, 2010, 2011, 2014)} for NACA0012 at $Re\sim77,000$
			\item <2-> Spontaneous pitch-oscillations which are not predicted by classical theory.
		\end{itemize}
		
	\end{itemize}
	
	\begin{tikzpicture}[remember picture,overlay]
	
	\node<1>(movie) at (page cs:0.30,0.30)
	{
		\includemedia[
		width=0.50\linewidth,
		keepaspectratio,
		activate=pageopen,
		playbutton=fancy,
		addresource=movies/naca_pitch.mp4,
		flashvars={source=movies/naca_pitch.mp4
			&loop=true}
		]{\includegraphics[width=1.00\textwidth]{naca_pitch00001}}{VPlayer.swf}
	};		
	
	\node <1>(image) at (page cs:0.8,0.35)
	{
		\includegraphics[width=.35\paperwidth]{poirel-setup.png}
	};
	\node <1>(image) at (page cs:0.90,0.55)
	{
		\text{\tiny Poirel (2008)}
	};
	
	\node <2->(image) at (page cs:0.5,0.30)
	{
		\includegraphics[width=.80\paperwidth]{poirel-re77k_response.png}
	};	
	\node <2->(image) at (page cs:0.85,0.13)
	{
		\text{\tiny Poirel (2008)}
	};
	
	\end{tikzpicture}
	
\end{frame}

%-------------------------------------------------------------------------------
\begin{frame}[t]{Introduction}
	{Laminar airfoils}
	
	\begin{itemize}
		\item <1-> Aeroelastic response of natural laminar wings.
		\begin{itemize}
			\item <1-> Forced small-amplitude pitch oscillations: \textbf{Nonlinear response}.
			\item <1-> \textit{Mai et al. (2011), Hebler et al. (2013), Lokatt (2017)}			
		\end{itemize}
	\end{itemize}
	
	\begin{tikzpicture}[remember picture,overlay]
	
	\node<1>(movie) at (page cs:0.20,0.30)
	{
		\includemedia[
		width=0.35\linewidth,
		keepaspectratio,
		activate=pageopen,
		playbutton=fancy,
		addresource=movies/saab_pitch.mp4,
		flashvars={source=movies/saab_pitch.mp4
			&loop=true}
		]{\includegraphics[width=1.00\textwidth]{saab_pitch00001}}{VPlayer.swf}
	};	
	
	\node <1>(image) at (page cs:0.70,0.33)
	{
		\includegraphics[width=.65\paperwidth]{unsteady_response}
	};
	\node <1>(image) at (page cs:0.39,0.17)
	{
		\text{\tiny Lokatt (2017)}
	};
	
	
	\end{tikzpicture}
	
\end{frame}

%-------------------------------------------------------------------------------

\begin{frame}[t]{Introduction}
	{Non-linear regime}
	
	\begin{itemize}
		\item Aeroelastic instability \\at transitional Re.
		\item Nonlinearity in \\NLF airfoils
	\end{itemize}
	
	\begin{tikzpicture}[remember picture,overlay]
	
	\node <1>(image) at (page cs:0.72,0.63)
	{
		\includegraphics[width=.60\paperwidth]{unsteady_response}
	};
	\node <1>(image) at (page cs:0.92,0.45)
	{
		\text{\tiny Lokatt (2017)}
	};	
	
	\node <1>(image) at (page cs:0.5,0.30)
	{
		\includegraphics[width=.70\paperwidth]{poirel-re77k_response.png}
	};	
	\node <1>(image) at (page cs:0.85,0.13)
	{
		\text{\tiny Poirel (2008)}
	};
	
	
	\end{tikzpicture}
	
\end{frame}

%-------------------------------------------------------------------------------
\begin{frame}[t]{Coupled FSI instability}
	{and motivation}
	
	\begin{itemize}
		\item <1-> Aeroelastic experiments for a NACA0012 at $Re_{c}=77,000$ \textit{(Poirel et al. 2008)}
		\begin{itemize}
			\item <1-> Instability not predicted by Classical theory.
			\item <1-> How does one predict the onset of instability?
		\end{itemize}
		\item <2-> Can a \textbf{global stability analysis} predict the instability onset?
	\end{itemize}
	
	\begin{tikzpicture}[remember picture,overlay]
	\node <1->(image) at (page cs:0.5,0.30)
	{
		\includegraphics[width=.70\paperwidth]{poirel-re77k_response.png}
	};	
	\node <1->(image) at (page cs:0.85,0.13)
	{
		\text{\tiny Poirel (2008)}
	};
	
	
	\end{tikzpicture}
	
\end{frame}

%%-------------------------------------------------------------------------------


\begin{frame}[t]{FSI linear stability}
	{Computational Challenge}
	
	\begin{itemize}
		\item <1-> Linearization is not straightforward.
		\begin{itemize}
			\item <1-> How does one deal with moving interfaces?
		\end{itemize}
	\end{itemize}
	
	\begin{tikzpicture}[remember picture,overlay]
	\node <1>(image) at (page cs:0.22,0.32)
	{
		\includegraphics[width=.6\paperwidth]{domain_equilibrium}
	};	
	
	\node <1>(image) at (page cs:0.75,0.32)
	{
		\includegraphics[width=.6\paperwidth]{domain_perturbed}
	};		
	\end{tikzpicture}
	
\end{frame}

%-------------------------------------------------------------------------------
\begin{frame}[t]{FSI Equations}
	{Linearization}
	
	\begin{itemize}
		\item Taylor-expansion based linearization.
		\begin{itemize}
			\item All variables defined on the equilibrium grid.
		\end{itemize}
	\end{itemize}
	
	\begin{tikzpicture}[remember picture,overlay]
	\node <1>(image) at (page cs:0.5,0.37)
	{
		\includegraphics[width=.80\paperwidth]{domain_equilibrium_definitions}
	};		
	
	\node <2->(image) at (page cs:0.5,0.37)
	{
		\includegraphics[width=.80\paperwidth]{domain_perturbed_definitions}
	};		
	%	\begin{scope}[very thick, blue]
	%		\draw (page cs:0.49,0.55) circle (3pt);
	%	\end{scope}
	%	\draw[blue, thick, ->] (page cs:0.49,0.55) -- (page cs:0.30,0.50);
	%	\node <1-> (text) at (page cs:0.20,0.50) {$\mathbf{(U^{0}},P)$, $\mathbf{(u'},p')$};
	%
	%   \onslide<2->{
	%	\begin{scope}[very thick, red]
	%		\draw (page cs:0.51,0.57) circle (3pt);
	%	\end{scope}	
	%	\draw[red, thick, ->] (page cs:0.51,0.57) -- (page cs:0.51,0.70);
	%	\node <1-> (text) at (page cs:0.51,0.73) {$\mathbf{U}=(\mathbf{U^{0}+\delta x\cdot\nabla U^{0}}) + (\mathbf{ u'}+\cancel{\mathbf{\delta x\cdot\nabla u'}})$};
	%	}	
	
	
	\end{tikzpicture}
	
\end{frame}

%-------------------------------------------------------------------------------

\begin{frame}[t]{Linearized FSI Equations}
	{Final expression}
	
%	\begin{eqnarray}
%	\footnotesize
%	\begin{split}
%	\left[U^{0}_{j}\frac{\partial U^{0}_{i}}{\partial x^{0}_{j}}   +
%	%
%	\frac{\partial P^{0}}{\partial x^{0}_{i}}	-
%	%
%	\frac{1}{Re}\frac{\partial^{2} U^{0}_{i}}{\partial x^{0}_{j}\partial x^{0}_{j}}
%	%
%	\right] +
%	%
%	\delta x_{l}\frac{\partial }{\partial x^{0}_{l}}\left[U^{0}_{j}\frac{\partial U^{0}_{i}}{\partial x^{0}_{j}}   +
%	%
%	\frac{\partial P^{0}}{\partial x^{0}_{i}}	-
%	%
%	\frac{1}{Re}\frac{\partial^{2} U^{0}_{i}}{\partial x^{0}_{j}\partial x^{0}_{j}}
%	%
%	\right] \\
%	\left[\frac{\partial u'_{i}}{\partial t}  +  U^{0}_{j}\frac{\partial u'_{i}}{\partial x^{0}_{j}} + u'_{j}\frac{\partial U^{0}_{i}}{\partial x^{0}_{j}}
%	+\frac{\partial p'}{\partial x^{0}_{i}} - \frac{1}{Re}\frac{\partial^{2}u'_{i}}{\partial x^{0}_{j}\partial x^{0}_{j}} \right] = 0
%	\end{split}\nonumber
%	\end{eqnarray}
	\normalsize
	\onslide<1->
	\begin{eqnarray}
	\underbrace{\left[\mathcal{NS}(\mathbf{U^{0}},P^{0}) + \mathbf{ \delta x}\cdot\nabla(\mathcal{NS}\mathbf{(U^{0}},P^{0} ))\right]}_{\text{Taylor expansion of Steady Navier--Stokes}} + \mathcal{LNS}(\mathbf{ u'},p',\mathbf{U^{0}}) &=& 0 \nonumber \\
	\onslide<2-> \mathcal{LNS}(\mathbf{ u'},p',\mathbf{U^{0}}) &=& 0 \nonumber \\
	\onslide<3-> \mathbf{ u'} + \mathbf{ \delta x}\cdot\nabla\mathbf{U^{0}}  &=& d\mathbf{(\delta x')}/dt 		\nonumber
	\end{eqnarray}

	\begin{tikzpicture}[remember picture,overlay]
	
	\end{tikzpicture}
	
\end{frame}

%-------------------------------------------------------------------------------

\begin{frame}[t]{Linear FSI}
	{Rigid-body motion validation}
	%	{\href{./movies/re750k_mpeg4.mp4}{\underline{Unsteady response}}}	
	
	\begin{itemize}
		\item <1-> Validation in a lot of cases.
	\end{itemize}
	\small
	%\onslide<4->{
	%\begin{tabular}{c |c }
	%		Case & Unstable frequency  \\
	%		\hline
	%		Cossu \& Morino &$3.803 \times10^{-3} \pm0.660$ \\
	%		Linear &$1.9895\times10^{-3}+0.6565$ \\
	%		Non-Linear &  $1.9925\times10^{-3}+0.6565$ \\
	%	   Arnoldi &  $1.9965\times10^{-3}\pm0.6565$ \\		
	%	\end{tabular}
	%}
	
	\begin{tikzpicture}[remember picture,overlay]
	
%	\node <1>(image) at (page cs:0.5,0.35)
%	{
%		\includegraphics[width=.60\paperwidth]{cossu230000}
%	};
	
	\node <1>(image) at (page cs:0.25,0.35)
	{
		\includegraphics[width=.49\paperwidth]{re23_evol}
	};
	
	\node <1>(image) at (page cs:0.75,0.35)
	{
		\includegraphics[width=.49\paperwidth]{re23_peaks}
	};
	
%	\node <3>(image) at (page cs:0.18,0.33)
%	{
%		\includegraphics[width=.35\paperwidth]{ellipse_vert_peaks}
%	};
%	\node <3>(image) at (page cs:0.18,0.14)
%	{
%		\text{\tiny Oscillating ellipse}
%	};
%	
%	\node <3>(image) at (page cs:0.50,0.33)
%	{
%		\includegraphics[width=.35\paperwidth]{ellipse30_peaks}
%	};
%	\node <3>(image) at (page cs:0.50,0.14)
%	{
%		\text{\tiny Rotating ellipse}
%	};
%	
%	\node <3>(image) at (page cs:0.82,0.33)
%	{
%		\includegraphics[width=.315\paperwidth]{ugis_log}
%	};
%	\node <3>(image) at (page cs:0.82,0.14)
%	{
%		\text{\tiny Rotating cylinder-splitter-plate}
%	};
	
	\end{tikzpicture}
	
\end{frame}

%-------------------------------------------------------------------------------

\begin{frame}[t]{Linear FSI}
	{Eigenvalue sensitivity}
	
	\begin{itemize}
		\item <1-> Structural sensitivity of the least stable eigenvalue for an oscillating cylinder at $Re=50$, $\rho_{s}/\rho_{f}=10$, close to resonance.
	\end{itemize}
	
	
	\begin{tikzpicture}[remember picture,overlay]
	
	\node <1->(image) at (page cs:0.25,0.52){Stationary Cylinder};
	\node <1->(image) at (page cs:0.25,0.32)
	{
		\includegraphics[width=.49\paperwidth]{wavemaker0000}
	};
	
%	\node <2>(image) at (page cs:0.75,0.52){$\omega/\omega_{n}=9.5$};
%	\node <2>(image) at (page cs:0.75,0.32)
%	{
%		\includegraphics[width=.49\paperwidth]{wavemaker0001}
%	};
	
	\node <1>(image) at (page cs:0.75,0.52){$\omega/\omega_{n}=0.97$};
	\node <1>(image) at (page cs:0.75,0.32)
	{
		\includegraphics[width=.49\paperwidth]{wavemaker0003}
	};
	
%	\node <4>(image) at (page cs:0.75,0.52){$\omega/\omega_{n}=0.19$};
%	\node <4>(image) at (page cs:0.75,0.32)
%	{
%		\includegraphics[width=.49\paperwidth]{wavemaker0004}
%	};
	
	\end{tikzpicture}
	
\end{frame}

%-------------------------------------------------------------------------------

\begin{frame}[t]{Aeroelastic instability}
	{Aeroelastic modes}
	
	\footnotesize
	\begin{tabular}{x{2cm} x{4.5cm} x{2cm} }
		$K_{s}$ (Nm/rad) & Linear Stability & Poirel (2008) \\
		\toprule
		$0.30$ & $50,000-60,000$& $55,000$ \\
		$0.15$ & $40,000-50,000$ & $48,000$ \\
		$0.00$ & $<40,000$ & $45,000$ \\
		\bottomrule
	\end{tabular}
	\vspace{5pt}
	\normalsize
	\begin{itemize}
		\item <2-> Frequency and instability mechanisms are decoupled.
	\end{itemize}
	
	
	\begin{tikzpicture}[remember picture,overlay]
	\node <1->(image) at (page cs:0.77,0.30)
	{
		\includegraphics[width=.45\paperwidth]{spectra_all_k2}
	};
	\node <1->(image) at (page cs:0.90,0.52)
	{
		\text{\footnotesize $K_{s}=0.30$}
	};
	
	\node <1->(image) at (page cs:0.27,0.27)
	{
		\includegraphics[width=.50\paperwidth]{poirel-pitch_amplitude_speed}
	};
	\node <1->(image) at (page cs:0.27,0.09)
	{
		\text{\tiny $U$ (m/s)}
	};
	
	
	\end{tikzpicture}
	
\end{frame}
%-------------------------------------------------------------------------------

\begin{frame}[t]{Aeroelasticity of Laminar airfoils}
	{Unsteady aerodynamic response}
	
	\begin{itemize}
		\item <1-> Wall-resolved LES.
		\item <1-> 2 Cases:
		\begin{itemize}
			\item <1-> $Re_{c}=100,000$ \& $Re_{c}=750,000$ (1.4 Billion grid points, 30k-40k cores @ KTH and Stuttgart).
			\item <1-> Unsteady simulations of small-amplitude forced pitching motion over several pitch cycles.			
		\end{itemize}	
	
	\end{itemize}
	
	\begin{tikzpicture}[remember picture,overlay]
	
%	\node <1>(image) at (page cs:0.5,0.64)
%{
%	\includegraphics[width=.60\paperwidth]{pitch_re750k0002}
%};
%\node <1>(image) at (page cs:0.2,0.75)
%{
%	{$\alpha=4.4^{\circ}$}
%};

\node <1->(image) at (page cs:0.5,0.28)
{
	\includegraphics[width=.60\paperwidth]{pitch_re750k0001}
};
%\node <1>(image) at (page cs:0.2,0.38)
%{
%	{$\alpha=2.4^{\circ}$}
%};
	
	\end{tikzpicture}
	
\end{frame}

%-------------------------------------------------------------------------------
\begin{frame}[t]{Pitching airfoil}
	{$Re_{c}=100,000$}
	
	\begin{itemize}
		\item <2-> Leading-edge LSB changes characteristic from convective to absolute instability.
	\end{itemize}
	
	\begin{tikzpicture}[remember picture,overlay]
	
	
	\node <1>(image1) at (page cs:0.3,0.45)
	{
		\includegraphics[width=0.40\paperwidth]{cf_time_surf100k_tr}
	};

	\node <1>(image1) at (page cs:0.30,0.78)
	{
		\text{\footnotesize Wall shear stress ($\tau_{w}$)}
	};


	\node <1>(image2) at (page cs:0.7,0.45)
	{
		\includegraphics[width=0.40\paperwidth]{cf_time_surf_grey100k}
	};
	\node <1>(image2) at (page cs:0.70,0.78)
	{
		\text{\footnotesize Separation}
	};		
	
	% Transition
	\node (A2) at (page cs: 0.13, 0.07) {};
	\node (B2) at (page cs: 0.17, 0.07) {};	
	\draw<1>[color=pink, line width=0.30mm] (A2) -- (B2);
	
	\node <1>(image2) at (page cs:0.25,0.07)
	{
		\text{\footnotesize Transition point}
	};		

	\node <2>(image3) at (page cs:0.3,0.35)
	{
		\includegraphics[width=0.40\paperwidth]{bubble_profiles}
	};	
	\node <2>(image3) at (page cs:0.75,0.35)
	{
		\includegraphics[width=0.40\paperwidth]{cusp_map_88}
	};

	\node <3>(image3) at (page cs:0.5,0.35)
	{
		\includegraphics[width=0.70\paperwidth]{w0_x_88}
	};	

	
	\end{tikzpicture}
	
\end{frame}

%-------------------------------------------------------------------------------

\begin{frame}[t]{Pitching airfoil}
	{$Re_{c}=750,000$}
	
	\begin{itemize}
		\item <2-> Steady and unsteady characteristics can be correlated using a simple phase-lag.
	\end{itemize}
	
	\begin{tikzpicture}[remember picture,overlay]
	
	
	\node <1>(image1) at (page cs:0.5,0.5)
	{
		\includegraphics[width=1.0\paperwidth]{cf_time_surf750k_trN9}
	};
	\node <1>(image1) at (page cs:0.23,0.78)
	{
		\text{\footnotesize Wall shear stress ($\tau_{w}$)}
	};	
	
	
	% Separation
	\node (A1) at (page cs: 0.13, 0.14) {};
	\node (B1) at (page cs: 0.17, 0.14) {};	
	\draw<1>[color=black, line width=0.30mm] (A1) -- (B1);
	
	\node <1>(image1) at (page cs:0.32,0.14)
	{
		\text{\footnotesize Separation contours ($\tau_{w}=0$)}
	};	
	
	% Transition
	\node (A2) at (page cs: 0.13, 0.17) {};
	\node (B2) at (page cs: 0.17, 0.17) {};	
	\draw<1>[color=red, line width=0.30mm] (A2) -- (B2);
	
	\node <1>(image2) at (page cs:0.25,0.17)
	{
		\text{\footnotesize Transition point}
	};	
	
%	\node <2-3>(image1) at (page cs:0.30,0.45)
%	{
%		\includegraphics[width=0.60\paperwidth]{cf_time_surf750k_trN9}
%	};	
	
	\node <2->(image3) at (page cs:0.25,0.40)
	{
		\includegraphics[width=.45\paperwidth]{750k_transition_alpha}
	};
	
	\node <2->(image4) at (page cs:0.75,0.40)
	{
		\includegraphics[width=.43\paperwidth]{750k_transition_alpha_e}
	};
	
%	\node (A) at (page cs: 0.75, 0.53) {};
%	\node (B) at (page cs: 0.75, 0.43) {};	
	
%	\draw<3>[implies-implies,double equal sign distance, line width=0.40mm] (A) -- (B);
	
%	\node <2>(image4) at (page cs:0.75,0.65)
%	{
%		{\footnotesize Phase-lag}
%	};
	\node <2->(image) at (page cs:0.25,0.65)
	{
		{\footnotesize Transition location}
	};
	
	\end{tikzpicture}
	
\end{frame}

%-------------------------------------------------------------------------------

\begin{frame}[t]{Pitching airfoil}
	{Non-linear aerodynamic force modeling}	
	
	\begin{itemize}
		\item <1-> Unsteady nonlinear behavior almost entirely explained by the non-linearity of the static characteristics.
	\end{itemize}
	
	
	\begin{tikzpicture}[remember picture,overlay]
	
%	\node <1>(image1) at (page cs:0.30,0.65)
%	{
%		\includegraphics[width=.35\paperwidth,height=0.25\paperwidth]{950k_time_plot_33_1}
%	};
%	\node <1>(image1) at (page cs:0.10,0.52)
%	{
%		\text{\scriptsize $k=0.01$}
%	};
%	
%	\node <1>(image2) at (page cs:0.75,0.65)
%	{
%		\includegraphics[width=.35\paperwidth,height=0.25\paperwidth]{950k_time_plot_33_3}
%	};
%	\node <1>(image1) at (page cs:0.56,0.52)
%	{
%		\text{\scriptsize $k=0.05$}
%	};
	
	\node <1>(image3) at (page cs:0.5,0.35)
	{
		\includegraphics[width=.60\paperwidth,height=0.45\paperwidth]{950k_time_plot_33_5}
	};
%	\node <1>(image1) at (page cs:0.10,0.16)
%	{
%		\text{\scriptsize $k=0.10$}
%	};
	
%	\node <1>(image4) at (page cs:0.75,0.29)
%	{
%		\includegraphics[width=.35\paperwidth,height=0.25\paperwidth]{950k_time_plot_33_9}
%	};
%	\node <1>(image1) at (page cs:0.56,0.16)
%	{
%		\text{\scriptsize $k=0.21$}
%	};
	
	\end{tikzpicture}
	
\end{frame}
%-------------------------------------------------------------------------------

\begin{frame}[t]{Other works}
	{local stability/numerics/implementation}	
	
	\begin{itemize}
		\item <1-> Control of bypass transition using spanwise wall-oscillations (MSc work).
		\begin{itemize}
			\item <1-> Examined the effect of wall-oscillations on the continuous spectrum of Orr-Sommerfeld Squire equations.
		\end{itemize}
		\item <2-> Analyzed the stability of the numerical method (Spectral-element-method).
		\begin{itemize}
			\item <2-> Implemented a stabilization procedure in the code (incorporated into the main repo).
		\end{itemize}
		\item <3-> Turbulent flows over a stationary wing. 
		\item <4-> All FSI implementations were done from scratch.
	\end{itemize}
	
	
	\begin{tikzpicture}[remember picture,overlay]
	
	\end{tikzpicture}
	
\end{frame}
%-------------------------------------------------------------------------------
